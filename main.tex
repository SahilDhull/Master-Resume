\documentclass[10pt]{extarticle}
%\documentclass[]{file}
\usepackage[margin=0.2in]{geometry}
\usepackage{romannum}
\usepackage[most]{tcolorbox}
\usepackage{enumitem}
\usepackage{hyperref}
\usepackage{tabularx}
\usepackage{multicol}
\usepackage{multirow}
\setlist[itemize]{noitemsep, topsep=0pt}
\addtolength{\parskip}{-1.5mm}
\tcbset{
    frame code={}
    center title,
    left=0pt,
    right=0pt,
    top=0pt,
    bottom=0pt,
    colback=gray!40,
    colframe=white,
    width=\dimexpr\textwidth\relax,
    enlarge left by=0mm,
    boxsep=3pt,
    arc=0pt,outer arc=0pt,
    }
\begin{document}
\begin{flushleft}
\noindent {\huge\textbf{Sahil Dhull}}
\end{flushleft}
Senior Undergraduate  \hfill\textbf{Email : }sahild@iitk.ac.in
\\Department of Computer Science and Engineering \hfill\textbf{Mobile : }+91-8360919817
\vspace{-6pt}
\\
\noindent\rule[0.5ex]{\linewidth}{1pt}
{\large \textbf{\begin{tcolorbox}\textsc{Academic Qualifications}\end{tcolorbox}}}
\begin{center}
\begin{tabular}{|p{2.5cm}|p{6.0cm}|p{8.5cm}|p{1.8cm}|}
\hline
\centering{\textbf{Year}} & \centering{\textbf{Degree/Certificate}} & \centering{\textbf{Institute}} & \textbf{CPI/$\%$}\\
\hline
\centering{2016} - Present & \centering{B.Tech} & \centering{Indian Institute of Technology, Kanpur} & 8.9/10\\
\hline
\centering{2016} & \centering{CBSE(\Romannum{12})} & \centering{Abhinav Public School, New Delhi} & 97.4$\%$\\
\hline
\centering{2014} & \centering{CBSE(\Romannum{10})} & \centering{DAV Public School, Kurukshetra} & 10/10\\
\hline
\end{tabular}
\end{center}
{\large \textbf{\begin{tcolorbox}\textsc{Scholastic Achievements}\end{tcolorbox}}}
\begin{itemize}
\renewcommand\labelitemi{-}
\item Secured \textbf{AIR 230} in \textbf{JEE Advanced 2016} among the 2 Lakh shortlisted candidates.
\item Secured \textbf{AIR 27} in \textbf{JEE Mains 2016} among the 15 Lakh candidates.
\item Cleared National Standard Examination in Physics (\textbf{NSEP}), National Standard Examination in Astronomy (\textbf{NSEA}) conducted by IAPT (Indian Association of Physics Teachers)
\item Awarded \textbf{KVPY 2014} fellowship, securing \textbf{AIR 46} (out of a total of about 40,000 students).
\item Participated in \textbf{North-Indian Science Fair 2012} at National Science Center, New Delhi.
\item Participated in \textbf{CBSE Regional Exhibition 2012}.
\item Awarded \textbf{National Talent Search Scholarship 2012} by National Council of Educational Research and Training.
\item Secured \textbf{1st} position in \textbf{State level Essay Writing Competition} by Govt. of Haryana in \textbf{2010}.
\end{itemize}
% \medskip
\vspace{2mm}
{\large \textbf{\begin{tcolorbox}\textsc{Internships}\end{tcolorbox}}}
\textbf{ADOBE} $|$ Big Data Experience Lab
\hfill\hfill\textit{May'19 - Jul'19}\\
\textbf{Research Intern} under Niyati Chhaya $|$ Bangalore, India
\begin{itemize}
\renewcommand\labelitemi{--}
\item The project aimed at "\textbf{Customizing web experiences using real-time user interaction data}".
\item Oversaw the conceptualization, data gathering, as well as the implementation for one of the DL models along with the complete implementation for user interaction modeling and capturing user information, and a live PoC (Proof of Concept).
% \item Gathered more than 100 ideas in the given problem area and then chose a hard research problem to work on for 2 months.
\item Hosted 2 surveys on Amazon Mechanical Turk (AMT) for data gathering using multiple websites.
Implemented mechanisms to capture user information in real time from web interactions using JS libraries, which was a non-trivial task.
% \item Implemented Deep Learning Models such as RBM and Auto-Encoders in Pytorch.
\item The live PoC, built using JS and Flask, shows both the final output as well as the inner workings of the models used.
\end{itemize}
% \medskip
\vspace{2mm}
{\large \textbf{\begin{tcolorbox}\textsc{Projects}\end{tcolorbox}}}
\textbf{GO to MIPS Compiler} 
\hfill\hfill\textit{Jan'19 - Apr'19}\\
Course CS335A [Compiler Design] under Prof. Amey Karkare
\hfill
\href{https://github.com/SahilDhull/ssac}{\texttt{Github}}
\begin{itemize}
\renewcommand\labelitemi{--}
\item Implemented a compiler in python for a subset of programs in GO language, targeting MIPS; using PLY framework.
\item Processed input code in 4 stages: Lexing, Parsing and Semantic Checks, Three-Address Code Generation, and Assembly Code (MIPS) translation.
\item Incorporated support for dynamic memory allocation, recursion, multi-dimension arrays, complex data types, multi-level pointers, multiple return types, functions with any number of parameters, and short variable declaration.
\end{itemize}
\vspace{2mm}
\textbf{English Premier League} 
\hfill\hfill\textit{Jan'19 - Apr'19}\\
Course CS315A [Database Management System] under Prof. Arnab Bhattacharya
\hfill
\href{https://drive.google.com/open?id=1jrmoUkKQsdSrfBb3sb7SXW7nT3tW8aFx}{\texttt{Report}}, 
\href{https://github.com/SahilDhull/EPL}{\texttt{Github}}
\begin{itemize}
\renewcommand\labelitemi{--}
\item Implemented a miniature version of English Premier League using LAMP stack in MySQL.
\item Added 20 triggers for real time updates of goal scores, stats and charts.
\item Ran a demo of a season showing substitutions, bookings, and goals in match, along-with the completion of the season.
\end{itemize}
\vspace{2mm}
% -------------------------------- to edit --------------------------------\\
\textbf{Motion Planning with Probabilistic Guarantees}
\hfill\hfill\textit{Jan'19 - Apr'19}\\
Course CS638A [Formal Methods in Robotics] under Prof. Indranil Saha
\hfill
\href{https://drive.google.com/open?id=1sBdSDRLIF9iEaQYQkGyBC9rGSwJIx3VO}{\texttt{Chapter}}
\begin{itemize}
\renewcommand\labelitemi{--}
\item Read 4 papers on the broad topic "Motion Planning with Probabilistic Guarantee".
\item Wrote a 20 page book chapter based on the papers.
\item Gave a 50 page presentation on the paper "Temporal Logic Motion Planning and Control With Probabilistic Satisfaction Guarantees".
\end{itemize}
\vspace{2mm}
\textbf{GemOS}
\hfill\hfill(\textit{Aug'18 - Nov'18})\\
Course CS330A [Operating Systems] under Prof. Debadatta Mishra
\hfill
\href{https://github.com/SahilDhull/Assign_OS}{\texttt{Github}}
\begin{itemize}
\renewcommand\labelitemi{--}
\item Extended various functionalities of GemOS operating system, by implementing 4 level page table radix tree, for new context.
\item Implemented system calls like expand, shrink, write, clone, sleep etc and exception handlers like Floating point and Page fault exception.
\item  Added signal handlers for SIGSEGV, SIGFPE, and SIGALRM signals and implemented scheduling using the round robin scheme.
\item Implemented Object Store functionalities for a basic filesystem with and without caching, using FUSE APIs. The permitted operations were read and write.
\end{itemize}
% -------------------------------------------  --------------------------------\\
\vspace{2mm}
\textbf{Painter and Genre Classification}
\hfill\hfill\textit{Aug'18 - Nov'18}\\
Course CS771A [Machine Learning] under Prof. Piyush Rai
\hfill
\href{https://drive.google.com/open?id=1z7EHuB3JJIuZX8bsxXrjmLi5mLkdwaBF}{\texttt{Report}}
\begin{itemize}
\renewcommand\labelitemi{--}
\item Used 2 approaches: Self-designed CNN and Classification after Feature extraction.
\item Constructed CNN using convolutions, ReLU activation and Max Pooling and, gained a maximum of 50$\%$ test accuracy.
\item Used VGG16 and ResNet50 for feature extraction and for classifcation, used Logistic regression, SVM (with RBF kernel) and K-Nearest Neighbour. Gained a maximum test accuracy of 75.2$\%$.
\end{itemize}
\vspace{2mm}
\textbf{Deliver It App}
\hfill\hfill\textit{Aug'18 - Nov'18}\\
Course CS252A [Computing Laboratory II] under Prof. Nisheeth Srivastava
\hfill
\href{https://github.com/SahilDhull/252_project}{\texttt{Github}}
\begin{itemize}
\renewcommand\labelitemi{--}
\item Designed a community based delivery app for Android and iOS using geolocation services on IONIC Framework.
\item Used Firebase Authentication Service for login system, Firebase Realtime Database for the backend, and Leaflet Maps for geolocation services.
\end{itemize}
\vspace{2mm}
\textbf{Fusion of Inertial Sensing IoT Devices} \hfill\hfill\textit{May'18 - July'18}\\
Course CS664A [IoT System Design] under Prof. Amey Karkare
\hfill
\href{https://drive.google.com/open?id=1Ke4T3OCpIyz1n92X255i7DAlqgbJZLXJ}{\texttt{Report}}
\begin{itemize}
\renewcommand\labelitemi{--}
\item Learnt about Hardware and Software aspects of OBLU (Multi IMU inertial sensing device).
\item Implemented a Fusion algorithm on Dual Foot-mounted Inertial Sensors data to reduce Systematic Heading Drift resulting in a more precise navigation system.
\item Plotted Real-time graphs showing Raw and Corrected trajectories to find out Drift and Distance Errors.
\end{itemize}
\vspace{2mm}
\textbf{SAE IIT Kanpur \textbar \ }Team Member \hfill\hfill\textit{Jan'17 - Feb'18}\\Faculty Advisor - Prof. Shantanu De, Department of Mechanical Engineering
\hfill
\href{https://www.iitk.ac.in/ame/sae/}{\texttt{Webpage}}
\begin{itemize}
\renewcommand\labelitemi{--}
\item Designed and fabricated a Formula race car (F-18) for Formula Bharat, a national collegiate design challenge.
\item Secured \textbf{$9^{th}$} position in Design Event, \textbf{$6^{th}$} in Business Plan and \textbf{$15^{th}$} position among 55 teams at Formula Bharat 2018.
\item Part of Powertrain subsystem; designed the sprocket on Solidworks and simulated (and optimized) the design on ANSYS.
\end{itemize}
\vspace{2mm}
% \newpage
{\large \textbf{\begin{tcolorbox}\textsc{Technical Skills}\end{tcolorbox}}}
\noindent\textbf{Programming Languages: }
\hfill
C, C++, Python, SQL, Mips
\\ Familiar: 
\hfill
Verilog, Bash, PHP, Javascript, Flask
\\
\textbf{Tools and Frameworks: }\\
Robotics: 
\hfill
MiniSAT, Z3 SMT solver, NuSMV Model Checker, LTLMoP, PRISM, UPPAAL, Spin\\
ML:
\hfill
Pytorch (NN); \texttt{Familiar:} Octave, R\\
Other:
\hfill
Flask, IONIC (App dev); \texttt{Familiar:} Pthreads , CUDA programming\\
Mechanical: 
\hfill
ANSYS (Structural), AutoCAD Fusion, Solidworks, Autodesk Inventor
\medskip
{\large \textbf{\begin{tcolorbox}\textsc{Relevant Coursework}\end{tcolorbox}}}
\begin{center}
\begin{tabular} {l l l l }
  \textbf{Computers:} & & & \\
  Compiler Design & Database Management System & Formal Methods in Robotics(A*) & Machine Learning \\
  Algorithms-II & Operating Systems & Theory of Computation & Computing Laboratory-I,II  \\
  IoT System Design & Data Structure and Algorithms & Computer Organization & Logic \\
  \textbf{Maths:} & & & \\
  Topology & Probability and Statistics & Discrete Mathematics & Abstract Algebra \\
  Linear Algebra and ODE & Introduction to Calculus & Complex Variables\\
  \textbf{Others:} & & & \\
  Introduction to Electronics & Introduction to Logic & Mechanics & Electrodynamics \\
\end{tabular}
\end{center}
%\medskip
%\end{itemize}
{\large \textbf{\begin{tcolorbox}\textsc{Extra-Curricular Activities}\end{tcolorbox}}}
\begin{itemize}
\item Bagged 3rd position in a Robotics event (Robotricks) in Takneek\textsc{\char13}16 (Inter-hall Technical Competition):
\\Built a robot to preform simple tasks like lifting blocks and detecting coloured strips.
\item Took part in Dance Competition in Galaxy (Inter-hall Cultural Event) held in Jan'17.
\item Played Lawn Tennis as Compulsory Physical Activity during July'16 to April'17.
\item Participation at State Level Swimming Competition in Haryana; Category - Under 14 boys.
\end{itemize}
\end{document}
