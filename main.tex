\documentclass[10pt]{extarticle}
%\documentclass[]{file}
\usepackage[margin=0.2in]{geometry}
\usepackage{romannum}
\usepackage[most]{tcolorbox}
\usepackage{enumitem}
\usepackage{hyperref}
\usepackage{tabularx}
\usepackage{multicol}
\usepackage{multirow}
\setlist[itemize]{noitemsep, topsep=0pt}
\addtolength{\parskip}{-1.5mm}
\tcbset{
    frame code={}
    center title,
    left=0pt,
    right=0pt,
    top=0pt,
    bottom=0pt,
    colback=gray!40,
    colframe=white,
    width=\dimexpr\textwidth\relax,
    enlarge left by=0mm,
    boxsep=3pt,
    arc=0pt,outer arc=0pt,
    }
\begin{document}
\begin{flushleft}
\noindent {\huge\textbf{Sahil Dhull}}
\hfill
\textbf{
\href{https://www.linkedin.com/in/sahildhull-25/}{LinkedIn}, \href{https://github.com/SahilDhull}{github}}
\end{flushleft}
Senior Undergraduate  \hfill\textbf{Email : }sahild@iitk.ac.in
\\Department of Computer Science and Engineering \hfill\textbf{Mobile : }+91-8360919817
\vspace{-6pt}
\\
\noindent\rule[0.5ex]{\linewidth}{1pt}
{\large \textbf{\begin{tcolorbox}\textsc{Academic Qualifications}\end{tcolorbox}}}
\begin{center}
\begin{tabular}{|p{2.5cm}|p{6.0cm}|p{8.5cm}|p{1.8cm}|}
\hline
\centering{\textbf{Year}} & \centering{\textbf{Degree/Certificate}} & \centering{\textbf{Institute}} & \textbf{CPI/$\%$}\\
\hline
\centering{2016} - Present & \centering{B.Tech} & \centering{Indian Institute of Technology, Kanpur} & 8.9/10\\
\hline
\centering{2016} & \centering{CBSE(\Romannum{12})} & \centering{Abhinav Public School, New Delhi} & 97.4$\%$\\
\hline
\centering{2014} & \centering{CBSE(\Romannum{10})} & \centering{DAV Public School, Kurukshetra} & 10/10\\
\hline
\end{tabular}
\end{center}
{\large \textbf{\begin{tcolorbox}\textsc{Scholastic Achievements}\end{tcolorbox}}}
\vspace{-1mm}
\begin{itemize}
\renewcommand\labelitemi{--}
\item Secured \textbf{AIR 230} in \textbf{JEE Advanced 2016} among 0.2 million shortlisted candidates
\item Secured \textbf{AIR 27} in \textbf{JEE Mains 2016} among 1.5 million candidates
\item Cleared National Standard Examination in Physics (\textbf{NSEP}), National Standard Examination in Astronomy (\textbf{NSEA}) conducted by IAPT (Indian Association of Physics Teachers)
\item Awarded \textbf{KVPY 2014} fellowship, securing \textbf{AIR 46} (out of a total of about 40,000 students)
\item Awarded \textbf{National Talent Search Scholarship 2012} by NCERT (National Council of Educational Research and Training)
\end{itemize}
% \medskip
\vspace{2mm}
{\large \textbf{\begin{tcolorbox}\textsc{Internship}\end{tcolorbox}}}
\textbf{ADOBE} $|$ Big Data Experience Lab, Bangalore
\hfill\hfill\textit{May'19 - Jul'19}
\\\textbf{Research Intern} under \textit{Dr. Niyati Chhaya}, Senior Research Scientist
\begin{itemize}
\renewcommand\labelitemi{--}
\item Objective : \textbf{Customizing web experiences using real-time user interaction data}
\item Led the data gathering and conceptualization for ”customizing web experience” part
\item Implemented models such as \textbf{RBMs} and \textbf{Autoencoders} in Pytorch, and designed a live PoC (\textbf{Proof of Concept})
\item Integrated the mechanisms to capture user information in \textbf{real time} from web interactions using JS libraries
\item Hosted 2 type of \textbf{surveys} on Amazon Mechanical Turk (\textbf{AMT}) for data gathering using multiple websites
% \item Introduced 2 deep learning based models, which  take in real time data, with the first one predicting the user profile based on the interactions and the second one leveraging this information to create a customized experience
% \item Spearheaded the data gathering and conceptualization for "customizing web experience" part followed by implementing models (such as RBMs and Auto-Encoders) in Pytorch and finally built a live PoC (Proof of Concept)
% and capturing user information.
% \item Oversaw the implementation of mechanisms to capture user information in real time from web interactions using JS libraries along with hosting 2 surveys on Amazon Mechanical Turk (AMT) for data gathering using multiple websites
\item The live PoC, built using \textbf{JS} and \textbf{Flask}, shows both the final output as well as the inner workings of the models used
\item Currently planning to file a \textbf{patent} for the approach and work done in the field
% \item Built a live PoC (Proof of Concept) using JS and Flask, which shows the final output and the inner workings of the models used.
\end{itemize}
% \medskip
\vspace{2mm}
{\large \textbf{\begin{tcolorbox}\textsc{Projects}\end{tcolorbox}}}
\textbf{GO to MIPS Compiler} 
\hfill\hfill\textit{Jan'19 - Apr'19}\\
Course CS335A [Compiler Design] under Prof. Amey Karkare
\hfill
\href{https://github.com/SahilDhull/ssac}{\texttt{Github}}
\begin{itemize}
\renewcommand\labelitemi{--}
\item Implemented a compiler in python for a subset of programs in \textbf{GO} language, targeting \textbf{MIPS}; using \textbf{PLY} framework
\item Processed input code in 4 stages: Lexing, Parsing and Semantic Checks, Three-Address Code Generation, and Assembly Code (MIPS) translation
\item Incorporated support for \textbf{dynamic memory} allocation, \textbf{recursion}, multi-dimension arrays, complex data types, multi-level pointers, multiple return types, functions with any number of parameters, and short variable declaration
\end{itemize}
\vspace{2mm}
\textbf{English Premier League} 
\hfill\hfill\textit{Jan'19 - Apr'19}\\
Course CS315A [Database Management System] under Prof. Arnab Bhattacharya
\hfill
\href{https://drive.google.com/open?id=1jrmoUkKQsdSrfBb3sb7SXW7nT3tW8aFx}{\texttt{Report}}, 
\href{https://github.com/SahilDhull/EPL}{\texttt{Github}}
\begin{itemize}
\renewcommand\labelitemi{--}
\item Implemented a miniature version of English Premier League using \textbf{LAMP} stack
\item Added 20+ triggers for real time updates of goal scores, stats and charts
\item Ran a demo of a season showing substitutions, bookings, and goals in match, along-with the completion of the season
\end{itemize}
\vspace{2mm}
% -------------------------------- to edit --------------------------------\\
\textbf{Motion Planning with Probabilistic Guarantees}
\hfill\hfill\textit{Jan'19 - Apr'19}\\
Course CS638A [Formal Methods in Robotics] under Prof. Indranil Saha
\hfill
\href{https://drive.google.com/open?id=1sBdSDRLIF9iEaQYQkGyBC9rGSwJIx3VO}{\texttt{Chapter}}
\begin{itemize}
\renewcommand\labelitemi{--}
\item Read 4 papers on the broad topic "Motion Planning with Probabilistic Guarantee"
\item Wrote a 20 page book chapter based on the papers
\item Gave a 50 page presentation on the paper "Temporal Logic Motion Planning and Control With Probabilistic Satisfaction Guarantees"
\end{itemize}
\vspace{2mm}
\textbf{GemOS}
\hfill\hfill\textit{Aug'18 - Nov'18}\\
Course CS330A [Operating Systems] under Prof. Debadatta Mishra
\hfill
\href{https://github.com/SahilDhull/Assign_OS}{\texttt{Github}}
\begin{itemize}
\renewcommand\labelitemi{--}
\item Extended various functionalities of GemOS operating system, by implementing 4 level page table radix tree, for new context
\item Implemented \textbf{system calls} (like expand, shrink, write, clone, sleep) and \textbf{exception handlers} like Floating point and Page fault
% exception
\item  Added \textbf{signal handlers} for SIGSEGV, SIGFPE, and SIGALRM signals and implemented scheduling using round robin scheme
\item Implemented \textbf{Object Store} functionalities for a basic filesystem with and without caching, using \textbf{FUSE} APIs
% with the permitted operations being read and write
\end{itemize}
% -------------------------------------------  --------------------------------\\
\vspace{2mm}
\textbf{Painter and Genre Classification}
\hfill\hfill\textit{Aug'18 - Nov'18}\\
Course CS771A [Machine Learning] under Prof. Piyush Rai
\hfill
\href{https://drive.google.com/open?id=1z7EHuB3JJIuZX8bsxXrjmLi5mLkdwaBF}{\texttt{Report}}
\begin{itemize}
\renewcommand\labelitemi{--}
\item Used 2 approaches: Self-designed Convolutional Neural Network (CNN) and Feature extraction with Classification
\item Constructed the CNN with \textbf{ReLU} activation and \textbf{Max Pooling} and, gained a maximum of \textbf{50$\%$} test accuracy
\item Used \textbf{VGG16} and \textbf{ResNet50} for feature extraction and for classifcation, used Logistic regression, SVM (with RBF kernel) and K-Nearest Neighbour, gaining a maximum test accuracy of \textbf{75.2$\%$}
\end{itemize}
\vspace{2mm}
\textbf{Deliver It App}
\hfill\hfill\textit{Aug'18 - Nov'18}\\
Course CS252A [Computing Laboratory II] under Prof. Nisheeth Srivastava
\hfill
\href{https://github.com/SahilDhull/252_project}{\texttt{Github}}
\begin{itemize}
\renewcommand\labelitemi{--}
\item Designed a community based delivery app for Android and iOS using geolocation services on \textbf{IONIC} Framework
\item Used Firebase Authentication Service for login system, Firebase Realtime Database for the backend, and Leaflet Maps for \textbf{geolocation} services
\end{itemize}
\vspace{2mm}
\textbf{Fusion of Inertial Sensing IoT Devices} \hfill\hfill\textit{May'18 - July'18}\\
Course CS664A [IoT System Design] under Prof. Amey Karkare
\hfill
\href{https://drive.google.com/open?id=1Ke4T3OCpIyz1n92X255i7DAlqgbJZLXJ}{\texttt{Report}}
\begin{itemize}
\renewcommand\labelitemi{--}
\item Learnt about Hardware and Software aspects of \textbf{OBLU} (Multi IMU inertial sensing device)
\item Implemented a \textbf{Fusion algorithm} on Dual Foot-mounted Inertial Sensors data to reduce Systematic Heading Drift resulting in a more precise navigation system
\item Plotted Real-time graphs showing Raw and Corrected trajectories to find out Drift and Distance Errors
\end{itemize}
\vspace{2mm}
\textbf{SAE IIT Kanpur \textbar \ }Team Member \hfill\hfill\textit{Jan'17 - Feb'18}\\Faculty Advisor - Prof. Shantanu De, Department of Mechanical Engineering
\hfill
\href{https://www.iitk.ac.in/ame/sae/}{\texttt{Webpage}}
\begin{itemize}
\renewcommand\labelitemi{--}
\item Designed and fabricated a Formula race car (F-18) for Formula Bharat 2018, a national collegiate design challenge
\item Secured $\mathbf{9^{th}}$ position in Design Event, $\mathbf{6^{th}}$ in Business Plan and $\mathbf{15^{th}}$ position among 55 teams from all over India
% short form --------------
% \item Part of Powertrain subsystem; designed the sprocket on Solidworks and simulated (and optimized) the design on ANSYS.
% long form ---------------
\item Part of \textbf{Powertrain} subsystem i.e. Engine and Drivetrain
\item Designed the Sprocket of the car in \textbf{Solidworks} and simulated and optimized its design on \textbf{ANSYS} for weight reduction, apart from manufacturing and assembling
% \item Part of Cost Report team, estimating and justifying the cost of the car
% -------------------------
\end{itemize}
\vspace{2mm}
% \newpage
{\large \textbf{\begin{tcolorbox}\textsc{Technical Skills}\end{tcolorbox}}}
\noindent\textbf{Programming Languages: }
\hfill
C, C++, Python, Bash, SQL, Mips
\\ Familiar: 
\hfill
MongoDB, Promela, Verilog, PHP, Javascript
\\
\textbf{Tools and Frameworks: }\\
Robotics: 
\hfill
MiniSAT, Z3 SMT solver, NuSMV Model Checker, LTLMoP, PRISM, UPPAAL, Spin\\
ML:
\hfill
Pytorch, Keras, Numpy, Scipy, Scikit-Learn; \texttt{Familiar:} MATLAB, R\\
Other:
\hfill
Flask, IONIC Framework; \texttt{Familiar:} Pthreads , CUDA programming\\
Mechanical: 
\hfill
ANSYS (Structural), AutoCAD Fusion, Solidworks, Autodesk Inventor
\medskip
{\large \textbf{\begin{tcolorbox}\textsc{Relevant Coursework}\end{tcolorbox}}}
\begin{center}
\begin{tabular} {l l l l }
  \textbf{Computers:} & & & \\
  Compiler Design & Computer-Aided Verification & Formal Methods in Robotics(A*) & Machine Learning \\
  Advanced Algorithms & Operating Systems & Theory of Computation & Database Management System  \\
  IoT System Design & Data Structure and Algorithms & Computer Organization & Computing Laboratory-I,II \\
  \textbf{Maths:} & & & \\
  Topology(A*) & Probability and Statistics & Discrete Mathematics & Abstract Algebra \\
  Linear Algebra and ODE & Introduction to Calculus & Complex Variables\\
  \textbf{Others:} & & & \\
  Introduction to Electronics & Introduction to Logic & Mechanics & Electrodynamics \\
\end{tabular}
\end{center}
\hfill A* - Grade for exceptional performance.
%\medskip
%\end{itemize}
{\large \textbf{\begin{tcolorbox}\textsc{Extra-Curricular Activities}\end{tcolorbox}}}
\begin{itemize}
\item Participated in \textbf{CBSE Regional Exhibition 2012} and \textbf{North-Indian Science Fair 2012} at National Science Center, New Delhi
\item Secured \textbf{1st} position in \textbf{State level Essay Writing Competition} by Govt. of Haryana in \textbf{2010}
\item Bagged 3rd position in a \textbf{Robotics} event (Robotricks) in Takneek\textsc{\char13}16 (Inter-hall Technical Competition):
\\Built a robot to preform simple tasks like lifting blocks and detecting coloured strips
\item Took part in \textbf{Dance Competition} in Galaxy (Inter-hall Cultural Event) held in Jan'17
\item Played \textbf{Lawn Tennis} as Compulsory Physical Activity during July'16 to April'17
\item Participation at \textbf{State Level Swimming} Competition in Haryana; Category - Under 14 boys
\end{itemize}
% \begin{center}
% \begin{tabular}{c | c}
%     \textbf{Technical} & \begin{tabular}{c}
%         Bagged 3rd position in a Robotics event (Robotricks) in Takneek\textsc{\char13}16 (Inter-hall Technical Competition):\\
%         Built a robot to preform simple tasks like lifting blocks and detecting coloured strips. 
%     \end{tabular}\\
%     \hline
%     \textbf{Cultural} & Took part in Dance Competition in Galaxy (Inter-hall Cultural Event) held in Jan'17.\\
%     \hline
%     \textbf{Sports} & 
% \end{tabular}
% \end{center}

\end{document}
